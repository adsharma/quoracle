\section{Read-Write Quorum Systems}
\NOTE[mwhittaker]{%
  These definitions are adapted from the quorum system definitions. We should
  double check that they all make sense or find the definitions in some paper.
}

Given a set $X = \set{x_1, \ldots, x_n}$, a \defword{read-write quorum system}
over $X$ is a pair of sets $Q = (R, W)$ of subsets of $X$ such that every $r
\in R$ intersects every $w \in W$. The elements $r \in R$ are called
\defword{read quorums}, and the elements $w \in W$ are called \defword{write
quorums}.

\TODO[mwhittaker]{%
  How do we define coteries for read-write quorums? Obviously, we don't want
  one read quorum being a strict subset of another read quorum, and we don't
  want a write quorum being a strict subset of another write quorum, but can a
  read quorum be a strict subset of a write quorum and vice-versa?
}

\TODO[mwhittaker]{%
  Once we define coteries, we can define domination.
}

Let $\sigma_R: R \to [0, 1]$ and $\sigma_W: W \to [0, 1]$ be a discrete
probability distribution over the read and write quorums of $Q$. We call
$\sigma$ a \defword{strategy}. Let $0 \leq p_r \leq 1$ be the probability of
performing a read and $p_w = 1 - p_r$ be the probability of performing a write.
We have the following definitions.
\begin{align*}
  l_{\sigma,p_r,p_w}(x)
    &\defeq p_r \cdot \parens*{\sum_{\setst{r \in R}{x \in r}} \sigma_R(r)} +
            p_w \cdot \parens*{\sum_{\setst{w \in W}{x \in w}} \sigma_W(w)} \\
  L_{\sigma,p_r,p_w}(Q) &\defeq \max_{x \in X} l_{\sigma,p_r,p_w}(x) \\
  L_{p_r,p_w}(Q)        &\defeq \min_\sigma L_{\sigma,p_r,p_w}(Q)
\end{align*}
$l_{\sigma,p_r,p_w}(x)$ is the load on $x$ given some strategy $\sigma$ and
some workload $p_r,p_w$. $L_{\sigma,p_r,p_w}(Q)$ is the load on most loaded
element $x$. $L{p_r,p_w}(Q)$ is the \defword{load} of the best possible
strategy.

The \defword{read resilience} or \defword{read fault tolerance} of a quorum
system $Q$ is the largest number $f_r$ such that for every subset $F \subseteq
X$ with $|F| = f_r$, there still exists some read quorum $r \in R$ such that $r
\cap F = \emptyset$.
%
The \defword{write resilience} or \defword{write fault tolerance} of a quorum
system $Q$ is the largest number $f_w$ such that for every subset $F \subseteq
X$ with $|F| = f_w$, there still exists some write quorum $w \in W$ such that $w
\cap F = \emptyset$.
%
The \defword{resilience} or \defword{fault tolerance} of a quorum system is the
minimum of its read resilience and write resilience. Intuitively, a quorum
system has read fault tolerance $f_r$ if we can fail an arbitrary set of $f_r$
elements and still have some read quorum left; a quorum system has write fault
tolerance $f_w$ if we can fail an arbitrary set of $f_w$ elements and still
have some write quorum left; and a quorum system has fault tolerance $f$ if we
can fail an arbitrary set of $f$ elements and still have some read quorum and
some write quorum left.
